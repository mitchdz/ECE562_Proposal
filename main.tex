\documentclass[12pt]{article}
\usepackage[margin=1in]{geometry}                % See geometry.pdf to learn the layout options. There are lots.
\geometry{letterpaper}                   % ... or a4paper or a5paper or ... 
%\geometry{landscape}                % Activate for for rotated page geometry
\usepackage[parfill]{parskip}    % Activate to begin paragraphs with an empty line rather than an indent

%%%%%%%%%%%%%%%%%%%%
\newcommand{\hide}[1]{}



\usepackage{natbib}
\usepackage{xcolor}
\usepackage{url}
\usepackage{hyperref}
\usepackage{mathtools}
\usepackage[utf8]{inputenc}
\usepackage{float}


\hide{
\usepackage{amscd}
\usepackage{amsfonts}
\usepackage{amsmath}
\usepackage{amssymb}
\usepackage{amsthm}
\usepackage{cases}		 
\usepackage{cutwin}
\usepackage{enumerate}
\usepackage{enumitem}
\usepackage{epstopdf}
\usepackage{graphicx}
\usepackage{ifthen}
\usepackage{lipsum}
\usepackage{mathrsfs}	
\usepackage{multimedia}
\usepackage{wrapfig}
}
\bibliographystyle{humanbio}


\usepackage[utf8]{inputenc}

\newcommand{\itemlist}[1]{\begin{itemize}#1\end{itemize}}
\newcommand{\enumlist}[1]{\begin{enumerate}#1\end{enumerate}}
\newcommand{\desclist}[1]{\begin{description}#1\end{description}}
\newcommand\tab[1][0.5cm]{\hspace*{#1}}

\newcommand{\Answer}[1]{\begin{quote}{\color{blue}#1}\end{quote}}
\newcommand{\AND}{\wedge}
\newcommand{\OR}{\vee}
\newcommand{\ra}{\rightarrow}
\newcommand{\lra}{\leftrightarrow}

\title {{\bf RISC V Based Drone Flight Controller} \\
\large{ECE 562 Project Proposal}}

\author{
Mitchell Dzurick - mitchdz@email.arizona.edu\\
Lena Voytek - dvoytek@email.arizona.edu\\
Amir Mohammad Asdagh Pour - asdaghpour@email.arizona.edu}
\date{2/17/2020}
\begin{document}

\maketitle
\begin{center}
\textbf{\url{https://github.com/mitchdz/ECE562}}
\end{center}


\tableofcontents 
\clearpage

\section{Names and email addresses of group members}
Mitchell Dzurick - mitchdz@email.arizona.edu\\
Lena Voytek - dvoytek@email.arizona.edu\\
Amir Mohammad Asdagh Pour - asdaghpour@email.arizona.edu

\section{Topic Description}
The goal of this project is to create a flight controller for a drone using a RISC V architecture. This includes purchasing and building a micro drone that will fashion a RISC V FPGA in order to create custom ISA subroutines for the drone. These custom subroutines will range from simple things such as move forward x feet to more complex things such as move to a certain GPS locaiton. The implementation of the more complex subroutines will get completed if time permits.

\section{Motivation}
The team taking on this project simply likes building and making drones. The project seems doable in a semester, but still offers a fun and challenging thing to solve. The goal here is not to make something super novel, but rather to have fun and learn a lot about RISC V ISA in the process.

\section{Related work}
\subsection{Worlds smallest autonomous rotor drone}
drone utilizing the GAP8 processor which is running 32-bit RISC-V to achieve a low power autonomous drone.

\subsection{This tiny drone with a tiny brain is smart enough to fly itself}
drone uses a mobile processor named GAP8. It packs eight processing cores optimized for running artificial intelligence applications.

\subsection{Development of Drone Capable of Autonomous Flight Using GPS}
An experimental drone was developed by equipping a
microcomputer of Raspberry Pi 2.0  and a GPS
sensor



\section{Proposed methodology}
The flight controller shall consist of an FPGA capable of booting verilog code. The codebase that is being looked at is ultraembedded's 32-bit RISC-V verilog implementation located at  \url{https://github.com/ultraembedded/riscv} which can be compiled using Vivado. This allows us to create our own ISA and accompanying instructions to do very specific tasks. Before this is done, a drone needs to be constructed and flight tested with a known-working flight controller such as the Naze32 flight controller. Once a build is done, the FPGA will be loaded with the RISCV core and linux will be put on the core. An attempt to control the drone via the FPGA GPIO (General Purpose Input Ouput) ports will be done. After being able to control the drone simply through Linux, the core will be modified for special subroutines. To evaluate the project, the 


\section{Proposed Timeline}
\iffalse
Proposed timelines. Provide a chart or table with the proposed timeline for work broken up
among the group members.
\fi

The following table shows a general outline for the project. These dates are of course subject to change, but are a guideline.

\begin{center}
\begin{tabular}{ |c|c|c| } 
 \hline
 Date & Task & Member \\ 
 \hline
 2/21 & Research Parts & Lena, Amir \\
 2/21 & Install software & Mitchell, Lena, Amir \\
 2/21 & Build RISC-V core & Mitchell, Lena \\
 2/31 & Purchase parts & Mitchell, Lena, Amir \\ 
 3/6 & Procure parts & Mitchell \\ 
 3/13 & Load Linux on FPGA & Lena, Mitchell\\
 3/13 & Base drone build & Mitchell, Amir\\
 3/31 & Get drone flying using Linux & Mitchell, Lena, Amir\\
 4/10 & Build custom ISA and instructions & Mitchell, Lena, Amir \\
 4/24 & Test all ISA implements & Mitchell, Lena, Amir \\
 \hline
\end{tabular}
\end{center}


\section{Anticipated Results}
The results of this project is to have a working drone that has a small set of instructions and a custom ISA for certain subroutines. 

\clearpage
\section{References}

Ultraembedded RISC-v \\
\tab\href{https://github.com/ultraembedded/riscv}{https://github.com/ultraembedded/riscv}

Worlds smallest autonomous rotor drone\\
\tab\href{https://riscv.org/2018/05/fast-company-article-this-tiny-drone-with-a-tiny-brain-is-smart-enough-to-fly-itself/}{https://riscv.org/2018/05/fast-company-article-this-tiny-drone-with-a-tiny-brain-is-smart-enough-to-fly-itself}

This tiny drone with a tiny brain is smart enough to fly itself \\
\tab\href{https://www.fastcompany.com/40575392/this-tiny-drone-with-a-tiny-brain-is-smart-enough-to-fly-itself}{https://www.fastcompany.com/40575392/this-tiny-drone-with-a-tiny-brain-is-smart-enough-to-fly-itself}


Development of Drone Capable of Autonomous Flight Using GPS \\
\tab\href{ttps://link.springer.com/article/10.1186/s40537-019-0214-3}{ttps://link.springer.com/article/10.1186/s40537-019-0214-3}


\end{document}
